\section*{Problem 3.1}
\subsection*{a}
Let $X$ denote the set of functions $f:\mathbb{R}^n\rightarrow\mathbb{C}$ such that
\begin{equation}
    \forall f \in X: ||f|| \equiv\int_{-\infty}^\infty dx^n f^\ast f = \int_{-\infty}^\infty dx^n|f|^2 < \infty
\end{equation}
Clearly, we inherit vector addition associativity, vector addition commutativity, and distributivity from the field $\mathbb{C}$ (where vector addition is pointwise addition in $\mathbb{C}$). To show the rest of the definitions, pick some $f,g \in X$ and $\phi \in \mathbb{C}$, then the following are equivalent
\begin{itemize}
    \item $X$ forms a vector space over $\mathbb{C}$
    \item $\phi f+g\in X$
    \item $||\phi f + g|| <\infty$
\end{itemize}
\begin{equation}
\begin{split}
    ||\phi f+g||
    &= \int_{-\infty}^\infty dx^n (\phi f+g)^\ast(\phi f + g)\\
    &= \int_{-\infty}^\infty dx^n (f^\ast\phi^\ast +g^\ast)(\phi f + g)\\
    &= \int_{-\infty}^\infty dx^n [\phi^\ast f^\ast(\phi f + g)+g^\ast(\phi f + g)]\\
    &= |\phi|^2||f||+||g||+\phi^\ast\int_{-\infty}^\infty dx^n f^\ast g +\left( \phi^\ast\int_{-\infty}^\infty dx^n f^\ast g \right)^\ast\\
    &= |\phi|^2||f||+||g||+2\Re\left[ \int_{-\infty}^\infty dx^n (\phi f)^\ast g\right] \\
\end{split}
\end{equation}
From the Schwartz Inequality
\begin{equation}
    \begin{split}
        \Re\left[ \int_{-\infty}^\infty dx^n (\phi f)^\ast g\right]
        &\leq |\int_{-\infty}^\infty dx^n (\phi f)^\ast g|^2 \leq \int_{-\infty}^\infty dx^n (\phi f)^\ast (\phi f)\int_{-\infty}^\infty dx^n g^\ast g\\
        &\leq |\phi|^2||f||||g||
    \end{split}
\end{equation}
Let $||f||=p$ and $||g||=q$, where $p,q \in \mathbb{R}$, then
\begin{equation}
    \begin{split}
        ||\phi f+g||
        &= |\phi|^2p+q++2\Re\left[ \int_{-\infty}^\infty dx^n (\phi f)^\ast g\right] \\
        &\leq |\phi|^2p+q+2|\phi|^2pq\\
        &<\infty
    \end{split}
\end{equation}
The set of normalized wave functions can not be a vector space because it lacks an additive identity.
\subsection*{b}
Let $V(\mathbb{C})$ be a vector space, $f,g \in V$ be arbitrary vectors, and $a,b\in\mathbb{C}$ be arbitrary scalars. Then $\braket{f,g}:V\times V \rightarrow\mathbb{C}$ is an inner product on the vector space $V$ iff

\begin{itemize}
    \item $\braket{f,g} = \braket{g,f}^\ast$
    \item $\braket{f,ag+bh} = a\braket{f,g}+b\braket{f,h}$\footnote{QM defines inner products as being linear in the second term, although really, it should be linear in the first term.}
    \item $\braket{f,f} \geq 0$ and $f=0 \Leftrightarrow \braket{f,f}=0$
\end{itemize}
We will show each of these properties for $\braket{f|g} \equiv \braket{g,f}$
\begin{equation}
    \begin{split}
        \braket{f|g}^\ast&=(\int_{-\infty}^\infty dx f^\ast g)^\ast\\
        &=\int_{-\infty}^\infty dx (f^\ast g)^\ast\\
        &=\int_{-\infty}^\infty dx g^\ast f\\
        &= \braket{g|f}
    \end{split}
\end{equation}
\begin{equation}
\begin{split}
    \braket{f|ag+bh} &= \int_{-\infty}^\infty dxf^\ast(ag+bh)\\
    &= \int_{-\infty}^\infty dx [af^\ast g+bf^\ast h]\\
    &= a\int_{-\infty}^\infty dx f^\ast g + b\int_{-\infty}^\infty dx f^\ast h\\
    &= a\braket{f|g}+b\braket{f|h}
\end{split}
\end{equation}
We know that $\forall x \in \mathbb{R}:|f(x)|^2 \geq 0$ so
\begin{equation}
    \begin{split}
        \braket{f|f} &= \int_{-\infty}^{\infty}dxf^\ast f\\
        &=\int_{-\infty}^\infty dx |f|^2\\
        &\geq 0
    \end{split}   
\end{equation}
The last condition is non-trivial. Let $f=0$ and $f'=\begin{cases}
    0 & x\neq0\\
    1
\end{cases}$ then $f'$ is clearly in $X$ and
\begin{equation}
    \begin{split}
        \braket{f|f} &= \int_{-\infty}^\infty dxf^\ast f = 0\\
        \braket{f'|f'}&= \int_{-\infty}^\infty dxf'^\ast f' = 0\\
    \end{split}
\end{equation}
However $f\neq f'$. So instead we define our inner product on the quotient space
$$X'\equiv X/\ker(||\cdot||)$$
The forward implication is already done, take $f=[0] = 0$ and refer to (8). Suppose then that $\braket{f|f}=0$
\begin{equation}
    \begin{split}
        \braket{f|f}
        &= \int_{-\infty}^\infty f^\ast f dx\\
        &= \int_{-\infty}^\infty |f|^2 dx\\
        &= 0 = ||f||
    \end{split}
\end{equation}
But since this is now unique we can say that $f = [0] = 0$.

\section*{Problem 3.3}
Let $h = f+ig$, then
\begin{equation}
    \begin{split}
        \braket{f+ig|\hat{Q}(f+ig)}
        &=\braket{f+ig|\hat{Q}f}+i\braket{f+ig|\hat{Q}g}\\
        &=(\braket{\hat{Q} f|f}+i\braket{\hat{Q}f|g})^\ast+i(\braket{\hat{Q} g|f}+i\braket{\hat{Q}g|g})^\ast\\
        &= \braket{f|\hat{Q}f} -i \braket{g|\hat{Q}f} +i \braket{f|\hat{Q}g} + \braket{g|\hat{Q}g}\\
        &= \braket{\Q}_f -i \braket{g|\hat{Q}f} +i \braket{f|\hat{Q}g} + \braket{\Q}_g\\
        \braket{\Q(f+ig)|f+ig}&= \braket{\Q}_f -i \braket{\Q g|f} +i \braket{\Q f|g} + \braket{\Q}_g\\
        \implies - \braket{\Q g|f} + \braket{\Q f|g} &= - \braket{ g|\Q f} + \braket{ f|\Q g}
    \end{split}
\end{equation}
If we let $h=f+g$, then
\begin{equation}
    \begin{split}
        \braket{f+g|\hat{Q}(f+g)}
        &=\braket{f+g|\hat{Q}f}+\braket{f+g|\hat{Q}g}\\
        &=(\braket{\hat{Q} f|f}+\braket{\hat{Q}f|g})^\ast+(\braket{\hat{Q} g|f}+\braket{\hat{Q}g|g})^\ast\\
        &= \braket{f|\hat{Q}f} + \braket{g|\hat{Q}f} + \braket{f|\hat{Q}g} + \braket{g|\hat{Q}g}\\
        &= \braket{\Q}_f + \braket{g|\hat{Q}f} + \braket{f|\hat{Q}g} + \braket{\Q}_g\\
        \braket{\Q(f+g)|f+g}&= \braket{\Q}_f + \braket{\Q g|f} + \braket{\Q f|g} + \braket{\Q}_g\\
        \implies  \braket{\Q g|f} + \braket{\Q f|g} &=  \braket{ g|\Q f} + \braket{ f|\Q g}
    \end{split}
\end{equation}
Adding the final result of equation of (10) and (11) together we get
\eq{
\bk{\Q f|g} = \bk{f|\Q g}
}
\section*{Problem 3.5}
\subsection*{a}
\begin{equation}
\begin{split}
    \braket{\psi | x\phi} &= \int_{-\infty}^\infty dx\psi^\ast x\phi\\
    &=\int_{-\infty}^\infty dx (x^\ast\psi)^\ast \phi\\
    &=\int_{-\infty}^\infty dx (x\psi)^\ast \phi\\
    &=\braket{x\psi|\phi}
\end{split}
\end{equation}
\begin{equation}
    \begin{split}
    \braket{\psi | i\phi} &= \int_{-\infty}^\infty dx\psi^\ast i\phi\\
    &=\int_{-\infty}^\infty dx (i^\ast\psi)^\ast \phi\\
    &=\int_{-\infty}^\infty dx (-i\psi)^\ast \phi\\
    &=\braket{-i\psi|\phi}
    \end{split}
\end{equation}
\begin{equation}
    \begin{split}
        \braket{\psi|\frac{d}{dx}\phi} &= \int_{-\infty}^\infty\psi^\ast\frac{d\phi}{dx}dx\\
        &=\psi^\ast\phi|_{-\infty}^\infty-\int_{-\infty}^\infty\frac{d\psi^\ast}{dx}\phi dx\\
        &=\braket{-\frac{d}{dx}\psi|\phi}
    \end{split}
\end{equation}

\subsection*{b}
\begin{equation}
    \begin{split}
        \braket{\psi|(\hat{Q}\hat{R})^\dagger|\phi} &= \braket{(\hat{Q}\hat{R})\psi|\phi}\\
        &=\braket{\hat{Q}(\hat{R}\psi)|\phi}\\
        &=\braket{\hat{R}\psi|\hat{Q}^\dagger\phi}\\
        &=\braket{\psi|\hat{R}^\dagger(\hat{Q}^\dagger\phi)}\\
        &=\braket{\psi|\hat{R}^\dagger\hat{Q}^\dagger|\phi}
    \end{split}
\end{equation}

\subsection*{c}
\begin{equation}
    \begin{split}
        \braket{\psi|a_+\phi} &= \braket{\psi|\frac{1}{\sqrt{2\hbar m \omega}}(-i\hat{p}+m\omega x)\phi}\\
        &= \frac{1}{\sqrt{2\hbar m \omega}} \left( \int_{-\infty}^\infty-\hbar\psi^\ast\frac{d\phi}{dx} dx + \int_{-\infty}^\infty m\omega \psi^\ast x\phi dx\right)\\
        &= \frac{1}{\sqrt{2\hbar m \omega}} \left( \hbar\int_{-\infty}^\infty\frac{d\psi^\ast}{dx}\phi dx + m\omega\int_{-\infty}^\infty (x\psi)^\ast \phi dx\right)\\
        &= \frac{1}{\sqrt{2\hbar m \omega}} \int_{-\infty}^\infty\left( \hbar\frac{d}{dx}+m\omega x\right)\psi^\ast\phi dx\\
        &= \frac{1}{\sqrt{2\hbar m \omega}} \int_{-\infty}^\infty\left( i\hat{p}+m\omega x\right)\psi^\ast\phi dx\\
        &=\braket{ \frac{1}{\sqrt{2\hbar m \omega}} \left( i\hat{p}+m\omega x\right)\psi|\phi}\\
        &=\braket{a_-\psi|\phi}
    \end{split}
\end{equation}
\section*{Problem 3.11}
The position ground state for the harmonic oscillator is
\begin{equation}
    \psi_0(x)=\left( \frac{m\omega}{\pi \hbar}\right)^\frac{1}{4}e^{-\frac{m\w}{2\h}x^2}
\end{equation}
In momentum space
\begin{equation}
    \begin{split}
        \phi_0(p) = \frac{1}{\sqrt{2\pi\h}}\br{\frac{m\w}{2\h}}^\frac{1}{4}\infi{x}\ex{-\frac{m\w}{2\h}x^2}\ex{-p\frac{p}{\h}x}
    \end{split}
\end{equation}
Completing the square and using an integral table yields
\begin{equation}
    \phi_0(x) = \frac{1}{\br{\pi\h m\w}^\frac{1}{4}}\ex{-\frac{p^2}{2\h m\w}}
\end{equation}
The energy associated with the ground state is
\begin{equation}
    E_0 = \frac{1}{2}\h\w
\end{equation}
And the classical momentum is $\n{p_0^2|2m}=E_0$ so the expected classical ground state momentum is
\eq{p_0=\sqrt{2m\w}}
Therefore the probability of measuring momentum not in the range of $\pm p_0$ is
\eq{\mathbb{P}&=\frac{1}{\sqrt{\pi\h m\w}}\br{\infi{p}\ex{-\frac{p^2}{\h m\w}}-\int_{p_0}^{p_0}dp\ex{-\frac{p^2}{\h m\w}}}\\
&=\frac{1}{\sqrt{\pi\h m\w}}\br{\sqrt{\pi\h m\w}-\sqrt{\pi  \hbar m \w}\text{ erf}\br{\frac{\sqrt{\pi \h m \w}}{\sqrt{\pi \h m \w}}}}\\
&=1-\textbf{erf}(1)\\
&\approx 0.16
}

\section*{Problem 3.32}
\subsection*{a}
Suppose $\braket{\Q} = a+ib$
\begin{equation}
    \begin{split}
        \braket{\Q}
        &=\braket{\psi|\Q\psi} \\
        &= \braket{-\Q\psi|\psi}\\
        &= -\braket{\Q\psi|\psi}\\
        &= -\braket{\psi|\Q\psi}^\ast\\
        &= -\braket{\Q}^\ast
    \end{split}
\end{equation}
Then $a+ib = -(a-ib) = -a + ib\implies a = 0$
\subsection*{b}
Suppose $\Q\ket{\la} = \la\ket{\la}$ and $\la = a+ib$, then
\eq{
\bk{\la|\Q\la} &= \la\\
\bk{\la|\Q\la} &= \bk{-\Q\la|\la}\\
&=-\bk{\Q\la|\la}\\
&=-\bk{\la|\Q\la}^\ast\\
&=-\la^\ast
}
It follows then that
\eq{
\la & = -\la^\ast\\
a+ib &= -a+ib\\
a &= -a
}
So we can deduce that $a=0$ and $\la$ must be purely imaginary.
\subsection*{c}
Let $\A$ and $\B$ be hermetian operators
\eq{
\bk{\p|[\A,\B]\phi} &= \bk{\p|[\A\B-\B\A]\phi}\\
&=\bk{[\A\B-\B\A]\da\p|\phi}\\
&=\bk{[(\A\B)\da-(\B\A)\da]\p|\phi}\\
&=\bk{[\B\da\A\da-\A\da\B\da]\p|\phi}\\ 
&=\bk{[\B\A-\A\B]\p|\phi}\\
&=\bk{-[\A,\B]\p|\phi}
}
$\implies$ the commutator is anti-hermetian
\subsection*{d}
Let $\A$ and $\B$ be anti-hermetian operators
\eq{
\bk{\p|[\A,\B]\phi} &= \bk{\p|[\A\B-\B\A]\phi}\\
&=\bk{[\A\B-\B\A]\da\p|\phi}\\
&=\bk{[(\A\B)\da-(\B\A)\da]\p|\phi}\\
&=\bk{[\B\da\A\da-\A\da\B\da]\p|\phi}\\ 
&=\bk{[(-\B)(-\A)-(-\A)(-\B)]\p|\phi}\\
&=\bk{-[\A,\B]\p|\phi}
}
$\implies$ the commutator is anti-hermetian

\subsection*{e}
Supposing that $Q$ is normal, we can write it's spectral decomposition as\footnote{The idea for this "proof" came from thinking about $Q$ as a normal operator on some finite dimensional $\mathcal{H}$. In the Eigen-basis, $Q$ is diagonal and we can split $Q$ into a real and imaginary diagonal matrix, $\A$ and $\B$, respectively. I was reading about how to write the spectral decomposotion of $Q$ in a more general Hilbert space in this
\hyperlink{https://math.stackexchange.com/questions/2639219/why-denote-the-spectral-decomposition-of-a-bounded-operator-as-an-integral}{SE post}.}
\eq{
Q &= \int\la dE(\la)\\
&= \int a(\la) dE(\la) + \int ib(\la)dE(\la)\\
&\equiv \A + \B
}
Next we'll show that $\A$ and $\B$ are hermetian and anti-hermetian, respectively
\eq{
\A\da &=  \br{\int a(\la) dE(\la)}\da\\
&=\int a(\la)\da dE(\la)\da\\
&= \int a(\la) dE(\la)\\
&= \A\\
\B\da &= \br{\int ib(\la) dE(\la)}\da\\
&=\int (ib(\la))\da dE(\la)\da\\
&=-\int b(\la) dE(\la)\\
&= -\B
}
This argument can be made rigorous using the limit definition. Taking the complex-conjugate of $Q$ yields
\eq{
Q\da &= \br{\A+\B}\da\\
&=\A\da+\B\da\\
&=\A-\B
}
So it follows that
\eq{
\A &= \n{1|2}\br{Q+Q\da}\\
\B &= \n{1|2}\br{Q-Q\da}
}

\section*{Problem 3.33}
\subsection*{a}
If we measure $a_1$ then immediately after measurement our system is in state $\psi_1$.
\subsection*{b}
The possible results of measuring $B$ are $b_1$ or $b_2$ with probability $\br{\frac{3}{25}}^2$ and $\br{\frac{4}{5}}^2$, respectively.
\subsection*{c}
We can write our system as a matrix, which is simple to invert
\begin{equation}
    \begin{split}
        5I_2\vec{\psi} &= \begin{pmatrix}
            3 & 4\\
            4 & -3
        \end{pmatrix}\vec{\phi}\\
        \vec{\phi}&=\frac{1}{5}\begin{pmatrix}
            -3 & -4\\
            -4 & 3
        \end{pmatrix}\vec{\psi}
    \end{split}
\end{equation}
So it follows that
\begin{equation}
    \begin{split}
        \phi_1 &= \frac{1}{5}\br{-3\psi_1-4\psi_2}\\
        \phi_2 &= \frac{1}{5}\br{-4\psi_1+3\psi_2}
    \end{split}
\end{equation}
If we are in state $\phi_1$ after measurement then we will be in state $\\psi_1$ with probability $\br{\frac{9}{5}}^2$. If we are in state $\phi_2$ after measurement then we will be in state $\psi_1$ with probability $\br{\frac{4}{5}}^2$.

From part (b), since these are independent measurements, the total probability of measuring $\psi_1$ is
\begin{equation}
    \br{\frac{9}{25}}^2    \br{\frac{9}{25}}^2+    \br{\frac{4}{16}}^2    \br{\frac{4}{16}}^2=\frac{337}{625}
\end{equation}