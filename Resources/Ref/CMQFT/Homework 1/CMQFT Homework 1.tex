\uchapter{Thermodynamics and Statistical Mechanics Review}
\usection{Problem 1.3.1}
Equation 1.14 is
\eq{
U(S,V) &= C ( \frac{e^{\frac{S}{nR}}}{V} )^\frac{2}{3}
}
Then, using the definition of temperature
\eq{
T &= (\frac{\partial S}{\partial U})^{-1} \\
\frac{\partial S}{\partial U} &= \partial_U nR\ln(\frac{VU^\frac{3}{2}}{C^\frac{3}{2}})\\
&= \frac{3nR}{2U}\\
\implies T &= \frac{2}{3nR}U
}
We now apply the Legendre transform to obtain the Helmholtz free energy potential
\eq{
F(T,V,N) &= U - TS\\
&= \frac{3nRT}{2} - T(nR\ln(\frac{VU^\frac{3}{2}}{C^\frac{3}{2}}))\\
&= \frac{3nRT}{2}(1-\frac{2}{3}\ln V-\ln U+\ln C)\\
&= \frac{3nRT}{2}((1+\ln C) - \ln(\frac{3nRT}{2})-\frac{2}{3}\ln V)
}
We can now take the partial derivatives with respect to $F$
\eq{
\frac{\partial F}{\partial T} &= \frac{3nR}{2}()
}
\eqe{
ppV0F &= - nRT/V
&= -P
}
\usection{Problem 1.6.1}
We want to find the partition function for the classical harmonic oscillator, where
\eq{
E(x,p) &= \frac{p^2}{2m} + \frac{1}{2}m\omega_0^2x^2
}
Then the partition function is, for a continuous $E$
\eq{
Z &= \int_{-\infty}^\infty \int_{-\infty}^\infty \exp(-\beta( \frac{p^2}{2m} + \frac{1}{2}m\omega_0^2x^2) ) dx dp
}
This is just two Gaussian integrals, which evaluate to
\eq{
Z &= \frac{2\pi}{\beta \omega_0}
}
\usection{Problem 1.6.2}
For the quantum harmonic oscillator the partition function becomes
\eq{
Z &= \sum_i e^{-\beta E_i}\\
&= \sum_n e^{-\beta \hbar \omega_0 (n + \frac{1}{2})}\\
&= e^{-\beta \frac{\hbar \omega_0}{2}}\sum_n \exp(-\beta \hbar \omega_0)^n\\
&= e^{-\beta \frac{\hbar \omega_0}{2}}(1-e^{-\beta \hbar \omega_0})^{-1}\\
&= \frac{1}{2}csch (\beta \frac{\hbar \omega_0}{2})
}
\usection{Problem 1.6.3}
The classical average energy is
\eq{
\braket{E_c} &= -\frac{1}{Z}\partial_\beta \ln Z\\
&= \frac{1}{\beta}
}
The quantum average energy is
\eq{
\braket{E_q} &= \frac{\hbar \omega_0}{2}\coth (\beta\frac{\hbar \omega_0}{2})
}
In the limit $k_B T \gg \hbar \omega_0$, we can Taylor expand the quantum energy
\eq{
\braket{E_q} &= \frac{1}{\beta}
}

\usection{Problem 1.8.1}
This problem just has us derive the Bose-Einstein and Fermi-Dirac distribution. The grand partition function is
\eqe{
\mathcal{Z} &= lsum N lsum i(N) exp(-beta(E(i)-mu N))
}
Only $0$ and $1$ Fermion can occupy a given energy level so
\eqe{
\mathcal{Z}_F &= 1 + exp(-beta(epsilon -mu))
}
Then the particle number average is
\eqe{
<N> &= 1/beta ppmu0 \ln(1 + exp(-beta(epsilon -mu)))
&= e^{-beta(epsilon-mu)}/1+e^{-beta(epsilon-mu)}
&= 1/e^{beta(epsilon-mu)}+1
}
For Bosons
\eqe{
\mathcal{Z}_B &= 1 + exp(-beta(epsilon - mu)) + exp(-2beta(epsilon - mu)) + ...
&= lsum k exp(-beta(epsilon - mu))^k
&= 1/1-e^{-beta(epsilon-mu)}
}
Then the particle number is
\eqe{
<N> &= 1/beta ppmu0 \ln( 1/1-e^{-beta(epsilon-mu)} )
&= e^{-beta(epsilon-mu)}/1-e^{-beta(epsilon-mu)}
&= 1/e^{beta(epsilon-mu)}-1
}