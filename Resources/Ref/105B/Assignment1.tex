\section{Thornton And Marion Problem 7.3}

% Problem drawing
\begin{figure}[h]
\centering
\begin{tikzpicture}[scale=1.5]
% Draw axis
\coordinate (origin) at (0,0);
\draw[dashed] (origin) -- node (axis) {} ($(origin)+(0,-2)$);

% Draw lower half of the circle
\draw (-2,0) arc (-180:0:2cm);

% Draw dashed line marking the radius
\draw[dashed] (origin) -- node[above] {R} (-20:2);

% Draw circle
\def\cradius{0.7}
\coordinate (a) at (60:\cradius-2);
\draw (a) circle (\cradius);
\draw[->] (origin) -- node (cpos) {} (a);

% Label normal
\draw[->] (a) -- node (cref) {} ($(a)+(-\cradius,0)$);

% label radius
\draw[dashed] (a) -- node[below] {$\rho$} node (cnorm) {} ($(a)+(-120:\cradius)$);

% Circle rotation angle
\pic [draw, ->, "$\theta$", angle eccentricity=1.5] {angle = cref--a--cnorm};

% Circle position angle
\pic [draw, ->, "$\psi$", angle eccentricity=1.5] {angle = cpos--origin--axis};

\end{tikzpicture}

\end{figure}

Rolling without slipping sets the constraint

\begin{equation}
    \theta \rho = \psi R
\end{equation}

The gravitational potential energy is

\begin{equation}
    U = mgR\left [1 - \sin(\psi)\right]
\end{equation}

The kinetic energy is

\section{Thornton and Marion Problem 7.9}

\section{Thornton and Marion Problem 7.26}

\section{}

\section{Thornton and Marion Problem 7.30}

\section{}