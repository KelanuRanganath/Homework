\usection{Laser Safety}
\subsection*{Part A}
Class I lasers emit a visible light less than 0.39mW, or are in an enclosure that requires a tool to open. Safe for eye exposure.

Class II lasers emit a visible beam of less than 1mW. Eye damage is likely to occur after a 1/4 second exposure.

Class III lasers emit a visible beam from 1mW up to 5mW. Any incidental exposure has some risk of eye damage.

Class IV lasers emit a visible beam of 600mW or more. Can cause burns and reflected or diffused light pose severe eye damage risk.

Optical density is the transmittance of a material on a log scale and is calculated by
\eqe{
OD &= -\log(T)
}

MPE stands for maximum permissible exposure, which is the maximum incident radiant exposure without causing harm.

\subsection*{Part B}
The latency in milli-seconds for the human blink reflex is $34.4ms$ on average. The MPE(E) for visible CW laser light is
\eqe{
MPE(E) &= \frac{1.8t^{0.75}}{t}
       &\approx 4.18 \frac{mW}{cm^2}
}

\subsection*{Part C}
Depending on the wavelength, rep rate, and peak intensity the most sensitive part of the eye is the retina followed by the cornea and lens.

\subsection*{Part D}
The most dangerous wavelengths are those that the human eye doesn't have a blink response and the ocular region.

\subsection*{Part E}
You should select laser goggles such that the potential incident power for the laser you're working with is attenuated to within the MPE. For example a $650nm$ CW laser with an average power of $1W$ would require goggles with at least an OD of $3$ at $650nm$. Of course, if the laser is pulsed, then you would want to calculate the peak power of the pulse and select the right OD for that power. 

\usection{Back-reflection From Optics}
The index of refraction at $635nm$ of air at STP is $1.000273$ and $1.4570$ for fused silica.
Assuming that the laser is normal to the optics, the reflectance is then
\eqe{
R_{s \lor p} &= ( n_1-n_2/n_1+n_2 ) ( n_1-n_2/n_1+n_2 )^\ast
	     &= 0.03455
}

So the power lost at the interface of the optic is $P_{incident}\cdot0.03455$.

The power transmitted through a lens on the first past is $P_iT^2$, due to back reflection inside the optics a smaller portion makes it through $PiT^2R^2$, this happens an infinite number of times and the total transmitted power can be written as a geometric series.
\eqe{
P_T &= P_iT^2 lsum k (R^2)^k
    &= P_i 1-R/1+R
\implies P_R &= P_i - P_T
	     &= P_i(1- 1-R/1+R )
	     &\approx P_i*0.06679
}
So approximately $7\%$ of the input power is lost. 


\usection{Photodiodes}
Assuming a CW beam. The responsivity of the DET36A is roughly $0.35A/W$. The current from the PD is
\eqe{
A &= V/R
  &= 4V/100k\Omega
  &= 0.04mA
\implies P &= W/0.35A \cdot 0.04mA
	   &= 0.11mW
}

The photodiode measures an energy average over time. 


\usection{Agreement}
I understand that the equipment in this laboratory class is both potentially dangerous and extremely expensive. I will therefore work diligently to ensure safety of my lab partner, the equipment, and myself. If I have any questions about how to proceed I will ask the TA or the professor BEFORE I act.

January 7th, 2025
