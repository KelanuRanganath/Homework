\usection{Crossed Polarizers}
The first polarizer filters the light such that $\psi = (a_H, 0)$, now suppose that the next filter's vertical axis is offset by an angle $\theta$ from the first polarizer.

The Jones matrix is then just the standard vertical polarizer in a frame rotated by $\theta$
$$
T = RT' = \begin{pmatrix}
    \cos \theta & -\sin \theta\\
    \sin \theta & \cos \theta
\end{pmatrix}
\begin{pmatrix}
    1 & 0\\
    0 & 0
\end{pmatrix}
= \begin{pmatrix}
    \cos \theta & 0 \\
    \sin \theta & 0 
\end{pmatrix}
$$
The new Jones vector is
$$
\psi_1 = a_H(\cos \theta, \sin \theta)
$$
Finally we go through the crossed polarizer, which removes the vertical compnoent, so
$$
\psi_2 = (0, a_H \sin \theta)
$$

\usection{Polarizing Beam Splitter}
A polarizing beam splitter separates lights by polarization. Suppose the incoming polarization is $\psi = (a_H, a_V)$, then the Jones matricies
$$
T_1 = \begin{pmatrix}
    1 & 0 \\
    0 & 0
\end{pmatrix}
$$
and
$$
T_2 = \begin{pmatrix}
    0 & 0 \\
    0 & 1
\end{pmatrix}
$$
The outgoing vectors are
$$
\psi_1 = (a_H,0)
$$
and
$$
\psi_2 = (0,a_V)
$$
, respectively.

\usection{Waveplate and a PBS}
The incoming signal is $\psi_0 = (0, a_V)$. The HWP Jones matrix is given by
$$
T_{HWP}(\theta) = \begin{pmatrix}
    \cos 2 \theta & \sin 2 \theta\\
    \sin 2 \theta & -\cos 2 \theta 
\end{pmatrix}
$$
The QWP Jones matrix is given by
$$
T_{QWP}(\theta) = \begin{pmatrix}
    \cos^2\theta-i\sin^2\theta & (1+i)\cos \theta \sin \theta\\
    (1+i)\cos \theta \sin \theta & \sin^2 \theta -i \cos^2 \theta 
\end{pmatrix}
$$
The PBS Jones matrix is given by
$$
T_{PBS}^{||V} = \begin{pmatrix}
    1 & 0\\
    0 & 0
\end{pmatrix}
$$
and
$$
T_{PBS}^{\perp V} = \begin{pmatrix}
    0 & 0\\
    0 & 1
\end{pmatrix}
$$

The transmitted power is then

\begin{table}[h]
    \centering
    \begin{tabular}{ccc}
         & QWP & HWP\\
        ||V & \sqrt{2}|\sin \theta \cos \theta | & |\sin 2 \theta |\\
        \perp V & \sin^4 \theta + \cos^4 \theta & |\cos 2 \theta|\\
    \end{tabular}
    \caption{Proportion, assuming unity input power, of transmitted power}
    \label{tab:my_label}
\end{table}

\usection{Optical Isolator}
The Jones matrix for a $\frac{\pi}{2}$ quarter-wave plate is
$$
T_{QWP}(\frac{\pi}{2}) = \begin{pmatrix}
    e^{-i\frac{\pi}{4}} & e^{i\frac{\pi}{4}}\\
    e^{i\frac{\pi}{4}} & e^{-i\frac{\pi}{4}}
\end{pmatrix}
$$
The Jones matrix then for a double pass is
$$
T_{QWP}^2(\frac{\pi}{2}) = \begin{pmatrix}
    0 & 1\\
    1 & 0
\end{pmatrix}
$$
Finally, if the input signal is then $\psi = (a_H, 0)$, then the output is
$$
\psi_{out} = \begin{pmatrix}
    0 & 1\\
    1 & 0
\end{pmatrix} \begin{pmatrix}
    a_H\\ 0
\end{pmatrix} = (0, a_H)
$$
If we pass this through the horizontal polarizer again, then
$$
\psi_{returned} = \begin{pmatrix}
    1 & 0\\
    0 & 0
\end{pmatrix}\begin{pmatrix}
    0\\ a_H
\end{pmatrix} = 0
$$
So we can see that no reflected light makes it back through the first polarizer.

\usection{Arbitrary Polarization}
First we will filter our light to start with a pure horizontal polarization, so
$$
\psi_i = (a_H, 0)
$$
Next we set the eccentricity with a half waveplate
$$
T_{HWP}(\theta + \frac{\phi}{2}) = \begin{pmatrix}
    \cos (2\theta + \phi) & \sin (2\theta + \phi)\\
    \sin (2\theta + \phi) & -\cos (2\theta + \phi)
\end{pmatrix}
$$
Where $\theta = \frac{1}{2}\arctan(\frac{b}{a})$. Next we use a quarter wave plate to make the polarization elliptical
$$
T_{QWP}(\Phi)T_{QWP}(\frac{\pi}{2}) = T_{QWP}(\Phi)\begin{pmatrix}
    1 & 0\\
    0 & e^{-i\frac{\pi}{2}}
\end{pmatrix}
$$
Finally, the output phase will be
$$
\psi_f = T_{QWP}(\Phi)\begin{pmatrix}
    1 & 0\\
    0 & e^{-i\frac{\pi}{2}}
\end{pmatrix}\begin{pmatrix}
    \cos (2\theta + \phi) & \sin (2\theta + \phi)\\
    \sin (2\theta + \phi) & -\cos (2\theta + \phi)
\end{pmatrix} \begin{pmatrix}
    a_H \\ 0
\end{pmatrix}
$$